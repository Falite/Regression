% Options for packages loaded elsewhere
\PassOptionsToPackage{unicode}{hyperref}
\PassOptionsToPackage{hyphens}{url}
%
\documentclass[
]{article}
\usepackage{amsmath,amssymb}
\usepackage{iftex}
\ifPDFTeX
  \usepackage[T1]{fontenc}
  \usepackage[utf8]{inputenc}
  \usepackage{textcomp} % provide euro and other symbols
\else % if luatex or xetex
  \usepackage{unicode-math} % this also loads fontspec
  \defaultfontfeatures{Scale=MatchLowercase}
  \defaultfontfeatures[\rmfamily]{Ligatures=TeX,Scale=1}
\fi
\usepackage{lmodern}
\ifPDFTeX\else
  % xetex/luatex font selection
\fi
% Use upquote if available, for straight quotes in verbatim environments
\IfFileExists{upquote.sty}{\usepackage{upquote}}{}
\IfFileExists{microtype.sty}{% use microtype if available
  \usepackage[]{microtype}
  \UseMicrotypeSet[protrusion]{basicmath} % disable protrusion for tt fonts
}{}
\makeatletter
\@ifundefined{KOMAClassName}{% if non-KOMA class
  \IfFileExists{parskip.sty}{%
    \usepackage{parskip}
  }{% else
    \setlength{\parindent}{0pt}
    \setlength{\parskip}{6pt plus 2pt minus 1pt}}
}{% if KOMA class
  \KOMAoptions{parskip=half}}
\makeatother
\usepackage{xcolor}
\usepackage[margin=1in]{geometry}
\usepackage{graphicx}
\makeatletter
\def\maxwidth{\ifdim\Gin@nat@width>\linewidth\linewidth\else\Gin@nat@width\fi}
\def\maxheight{\ifdim\Gin@nat@height>\textheight\textheight\else\Gin@nat@height\fi}
\makeatother
% Scale images if necessary, so that they will not overflow the page
% margins by default, and it is still possible to overwrite the defaults
% using explicit options in \includegraphics[width, height, ...]{}
\setkeys{Gin}{width=\maxwidth,height=\maxheight,keepaspectratio}
% Set default figure placement to htbp
\makeatletter
\def\fps@figure{htbp}
\makeatother
\setlength{\emergencystretch}{3em} % prevent overfull lines
\providecommand{\tightlist}{%
  \setlength{\itemsep}{0pt}\setlength{\parskip}{0pt}}
\setcounter{secnumdepth}{-\maxdimen} % remove section numbering
\ifLuaTeX
  \usepackage{selnolig}  % disable illegal ligatures
\fi
\usepackage{bookmark}
\IfFileExists{xurl.sty}{\usepackage{xurl}}{} % add URL line breaks if available
\urlstyle{same}
\hypersetup{
  pdftitle={notes},
  pdfauthor={Abdoullatif},
  hidelinks,
  pdfcreator={LaTeX via pandoc}}

\title{notes}
\author{Abdoullatif}
\date{2024-05-23}

\begin{document}
\maketitle

\section{TP01 OMARJEE Abdoullatif}\label{tp01-omarjee-abdoullatif}

l'objectif de ce TP est de se familiariser avec les outils de régression
linéaire sur R.\\
Ce compte rendu a été réalisé de sorte à pouvoir retourner sur les
scripts R et de les rééxécuter à souhait , au besoin.\\
De ce fait, la démultiplication des scripts R ont été volontairement
choisies au profit de la lisibilité du markdown, de sorte à ce que le
lecteur puisse se réapproprier le projet sans s'y perdre.\\

\subsection{Exercice 1}\label{exercice-1}

selon les informations prises sur
\href{https://www.kaggle.com/datasets/uciml/pima-indians-diabetes-database}{Kaggle}
ce database consiste en une étude réalisée sur 768 femmes appartenantes
à la communauté des Pima, réalisé par la
\href{https://www.niddk.nih.gov/}{NIH} .\\
il y a 9 variables différentes décrivant l'état médical des sujets, la
variable {pregnant} indiquant le nombre de grossesses qu'a réalisé
chaque sujet et les variables {glucose},

\subsubsection{question 1 :}\label{question-1}

on remarque qu'il y a plusieurs valeurs anormales : par exemple, le
{bmi} ( IMC ) ne peut pas être de 0 comme vu sur
\href{données.R}{données} .\\
on observe les mêmes problèmes concernant la variable {glucose} ,
{diastolic} , {insulin} , {triceps}.\\
Pour aller plus loin, on se munit d'un dataframe {pimanull} dans lequel
on a stocké les observations de {pima} qui posent problème.\\
On a ensuite affiché le nombre d'observations de {pima} et de
{pimanull}.On remarque que les données qui posent problème sont non
négligeables au vu des données, et représentent presque la moitié.\\
Notre choix a donc été de séparer les données problématiques pour ne
conserver que les données cohérentes.\\
On a donc corrigé les données, en stockant les résultats dans {pima1}\\

\subsubsection{question 2 :}\label{question-2}

Notre choix a donc été de séparer les données problématiques pour ne
conserver que les données cohérentes.\\
On a donc corrigé les données, en stockant les résultats dans {pima1}\\
on se sépare des données problématiques et on les stocke dans
l'environnement sous le nom de {pima1} , on a utilisé la commande
{which} en conséquence.\\
en ressortant le résumé du dataframe, on remarque que les variables sont
beaucoup plus cohérentes.

\subsubsection{question 3 :}\label{question-3}

Dans \href{pression_sanguine.R}{le script dédié à la pression sanguine},
on utilise la fonction {hist} pour afficher les données liées à la
pression sanguine diastolique , appelée {diastolic} dans le dataframe
{pima1} que nous utiliserons pour toute la suite de l'étude.\\
On a ensuite utilisé la fonction {[}{]}

\end{document}
